\documentclass[norsk]{beamer}

\usepackage[latin1]{inputenc}
\usepackage[T1]{fontenc}
\usepackage{babel,textcomp,pdfpages}
\usepackage{graphicx}
\usepackage{listings}
\usepackage{color}
\usepackage{tikz}
\usetikzlibrary{shapes,arrows}

\definecolor{dkgreen}{rgb}{0,0.6,0}
\definecolor{gray}{rgb}{0.5,0.5,0.5}
\definecolor{mauve}{rgb}{0.58,0,0.82}

\setbeamercolor{upcol}{fg=white,bg=blue!70}
\setbeamercolor{lowcol}{fg=white,bg=blue!40}

% "define" Scala
\lstdefinelanguage{scala}{
  morekeywords={abstract,case,catch,class,def,
    do,else,extends,false,final,finally,
    for,if,implicit,import,match,mixin,
    new,null,object,override,package,
    private,protected,requires,return,sealed,
    super,this,throw,trait,true,try,
    type,val,var,while,with,yield},
  otherkeywords={=>,<-,<\%,<:,>:,\#,@},
  sensitive=true,
  morecomment=[l]{//},
  morecomment=[n]{/*}{*/},
  morestring=[b]",
  morestring=[b]',
  morestring=[b]"""
}

% Default settings for code listings
\lstset{frame=tb,
  language=scala,
  aboveskip=3mm,
  belowskip=3mm,
  showstringspaces=false,
  columns=flexible,
  basicstyle={\small\ttfamily},
  numbers=none,
  numberstyle=\tiny\color{gray},
  keywordstyle=\color{blue},
  commentstyle=\color{dkgreen},
  stringstyle=\color{mauve},
  frame=none,
  breaklines=true,
  breakatwhitespace=true
  tabsize=3
}

\usetheme{Pittsburgh}
\setbeamertemplate{navigation symbols}{}

\title{Scala -- Introduksjon}
\author{Eivind Barstad Waaler}
\institute{BEKK/UiO}
\date{Fagdag -- 20. november 2009}

\begin{document}

\frame{\titlepage}

\section{Introforedrag}
\subsection{Introduksjon til Scala}
\begin{frame}{Hva er Scala?}
  \begin{itemize}
    \item Objektorientering + Funksjonell programmering
    \item Statisk/sterkt typet -- kompilert
    \item Java bytecode $\rightarrow$ JVM
    \item Kort historie:
      \begin{itemize}
        \item 1995 -- Pizza $\rightarrow$ GJ $\rightarrow$ javac \& Java generics
        \item 2001 -- Scala design p�begynt av Martin Odersky (EPFL)
        \item 2003 -- Scala versjon 1.0
        \item 2006 -- Scala versjon 2.0
        \item I dag -- Scala versjon 2.7.7 $\rightarrow$ 2.8
      \end{itemize}
  \end{itemize}
\end{frame}

\tikzstyle{block} = [rectangle, draw, fill=blue!20, text width=5em, text centered, rounded corners, minimum height=4em]
\tikzstyle{line} = [draw, -latex']
\tikzstyle{cloud} = [draw, ellipse,fill=red!20, node distance=2.5cm, minimum height=2em]
\begin{frame}[fragile]
  \frametitle{Scala \& Java}
  \begin{figure}
  \begin{tikzpicture}[node distance = 2cm, auto]
    % Place nodes
    \node [cloud] (class) {Java Bytecode (.class)};
    \node [block, left of=class, above of=class] (scalacompile) {Scala Kompilator (scalac)};
    \node [cloud, left of=scalacompile] (scala) {.scala};
    \node [block, right of=class, above of=class] (javacompile) {Java Kompilator (javac)};
    \node [cloud, right of=javacompile] (java) {.java};
    \node [block, below of=class] (jvm) {Java Virtual Machine};
    \node [block, below of=scala] (scalalib) {Scala API (.jar)};
    % Draw edges
    \path [line] (scalacompile) -- (class);
    \path [line] (javacompile) -- (class);
    \path [line] (class) -- (jvm);
    \path [line] (scala) -- (scalacompile);
    \path [line] (java) -- (javacompile);
    \path [dotted, line] (scalalib) -- (jvm);
  \end{tikzpicture}
  \end{figure}
\end{frame}

\subsection{Eksempler}

\begin{frame}{REPL -- Read Eval Print Loop}
  \begin{itemize}
    \item Scala interpreter
    \item Skriv bare \texttt{scala} p� kommandolinjen
    \item Maven prosjekt: \texttt{mvn scala:console}
    \item Scala 2.8 $\rightarrow$ Command completion + Interactive debugging
  \end{itemize}
  \vfill
  \begin{beamerboxesrounded}[upper=upcol,lower=lowcol,shadow=true]{Demo:}
    REPL med noen enkle uttrykk..
  \end{beamerboxesrounded}
\end{frame}

\begin{frame}[fragile]
  \frametitle{\texttt{var} og \texttt{val}}
  \begin{lstlisting}
// val - immutable (kan ikke endre verdi)
val constant = 5
constant *= 2 // error: reassignment to val

// var - mutable (variabel verdi)
var tall = 3
tall *= 2 // tall = tall * 2 = 3 * 2 = 6
  \end{lstlisting}
\end{frame}

\begin{frame}[fragile]
  \frametitle{Syntaks Basis}
  \begin{lstlisting}
// Typen skrives bak
val navn: String = "Eivind"
val alder: Int = 33

// Eller oppdages automatisk av kompilator (inference)
val firma = "Bekk"
val ansatte = 210

// Metoder med type og returverdi
def storBokstav(s: String): String = {
  s.toUpperCase
}
// Returtype kan ofte droppes, samt klammene:
def storBokstav(s: String) = s.toUpperCase
// Metoder uten returverdi (Unit -> void i Java)
def skriv(s: String) { println(s) }
  \end{lstlisting}
\end{frame}

\begin{frame}[fragile]
  \frametitle{Klasser og Objekter}
  \begin{lstlisting}
// Klasser omtrent som i Java
class Person(val navn: String, val alder: Int) {
  def erGammel = alder > 30
}

val p = new Person("eivind", 33)
p.erGammel // true :(

// Objekter er singletons
object Person {
  def apply(navn: String, alder: Int) =
    new Person(navn, alder)
}

val p2 = Person("erlend", 27)
  \end{lstlisting}
\end{frame}

\begin{frame}[fragile]
  \frametitle{Traits}
  \begin{lstlisting}
// Traits ligner klasser
trait Hilse {
  def siHei(s: String) = "Heihei " + s
}
trait Forlate {
  def morna(s: String) = "Hadetbra, " + s
}

// Funker som mixins -> Java aop?
class FinPerson(n: String, a: Int)
  extends Person(n, a) with Hilse with Forlate

val fin = new FinPerson("oc", 27)
fin.erGammel // false :)
fin.siHei("Kari") // "Heihei Kari"
  \end{lstlisting}
\end{frame}

\begin{frame}[fragile]
  \frametitle{H�yere-ordens Funksjoner}
  \begin{lstlisting}
// Funksjon som variabel
val doble = (i: Int) => i * 2

// Funksjon som parameter
val tall = Array(1, 2, 3, 4, 5)
val dobleTall = tall.map(doble)

// Funksjon som returverdi (og Closure faktisk)
val grense = (g: Int) => (i: Int) => i < g
val grense10 = grense(10)
grense(9) // true
grense(10) // false
  \end{lstlisting}
\end{frame}

\begin{frame}[fragile]
  \frametitle{Anonyme Metoder}
  \begin{lstlisting}
val tall = Array(5, 4, 3, 2, 1)

// Finn partall - anonym funksjon
val partall = tall.filter((i: Int) => i % 2 == 0)
// Eller med _ som anonym parameter
val partall = tall.filter(_ % 2 == 0)

// Eksempel fra forrige slide
val dobleTall = tall.map(_ * 2)

// Eller personklassen v�r
val personer = List(Person("eivind", 33),
  Person("erlend", 27), Person("oc", 27))
val (gamle, unge) = personer.partition(_.erGammel)
// Scala har tuples :)
  \end{lstlisting}
\end{frame}

\begin{frame}[fragile]
  \frametitle{\texttt{if}, \texttt{else} og \texttt{while}}
  \begin{lstlisting}
val p = Person("eivind", 33)

// if/else som Java
if (p.erGammel) println(p.navn + " er gammel")

// if/else returnerer verdi - ingen ternary op
val alder = if (p.erGammel) "gammel" else "ung"

// while/do-while returnerer ingen verdi - Unit
while (true) println("evig l�kke..")

do { println("evig l�kke..") } while (true)

// FP liker ikke while-l�kker...
  \end{lstlisting}
\end{frame}

\begin{frame}[fragile]
  \frametitle{For Comprehensions}
  \begin{lstlisting}
// Man bruker en eller flere "generators"
for (i <- 0 to 10) println(i)
for (i <- 0 to 10; j <- 0 to 10) println(i + "," + j)

// Kan gi en verdi - som Range
val doble = for (i <- 0 to 10) yield (i * 2)

// Har ogs� "filtering" og "binding"
val personer = List(Person("eivind", 33), ...
// Med {} slipper man ; mellom uttrykkene
val gammelE = for {
  person <- personer // generator
  if (person.erGammel) // filter
  navn = person.navn // binding
  if (navn startsWith "e") // filter
} yield { navn.toUpperCase }
  
  \end{lstlisting}
\end{frame}

\begin{frame}[fragile]
  \frametitle{Pattern Matching}
  \begin{lstlisting}
// Kan minne om en veldig kraftig switch
def sjekk(obj: Any) = obj match {
  case s:String => "String: " + s
  case 42 => "The answer 42"
  case i: Int => "Some other int: " + i
  case (s:String, i:Int) => "SI tuple: " + s + " " + i
  case _ => "Something else.. " + obj
}

sjekk("eivind")
sjekk(42)
sjekk(33)
sjekk("eivind", 33)
sjekk(42.1)
sjekk(new Person("eivind", 33))
  \end{lstlisting}
\end{frame}

\subsection{Ressurser}
\frame{
  \frametitle{Ressurser}
  \begin{beamerboxesrounded}[upper=upcol,lower=lowcol,shadow=true]{Scala Website:}
    \url{http://www.scala-lang.org/}
  \end{beamerboxesrounded}
  \vfill
  \begin{beamerboxesrounded}[upper=upcol,lower=lowcol,shadow=true]{Programming in Scala (ligger i Confluence):}
    \url{https://nanna.bekk.no/confluence/download/attachments/5082351/ProgInScala1EdV6.pdf}
  \end{beamerboxesrounded}
  \vfill
  \begin{beamerboxesrounded}[upper=upcol,lower=lowcol,shadow=true]{ScalaDoc API:}
    \url{http://www.scala-lang.org/docu/files/api/index.html}
  \end{beamerboxesrounded}
}

\end{document}